\documentclass[letterpaper]{article} 
\usepackage[utf8]{inputenc}
\linespread{0.85}
\usepackage[T1]{fontenc}
\usepackage{amsmath}
\usepackage{amsfonts}
\usepackage{amssymb}
\usepackage{array}
\usepackage{booktabs}
\usepackage{hyperref}
\usepackage[version=4]{mhchem}
\usepackage[dvipsnames]{xcolor}
\colorlet{LightRubineRed}{RubineRed!70}
\colorlet{Mycolor1}{green!10!orange}
\definecolor{Mycolor2}{HTML}{00F9DE}
\usepackage{graphicx}
\usepackage{amsmath}
\usepackage{graphicx}
\usepackage{capt-of}
\usepackage{lipsum}
\usepackage{fancyvrb}
\usepackage{tabularx}
\usepackage{listings}
\usepackage[export]{adjustbox}
\graphicspath{ {./images/} }
\usepackage[utf8]{inputenc}
\usepackage[english]{babel}
\usepackage{float}
\usepackage{lipsum}
\usepackage{graphicx}
\usepackage{float}
\usepackage[margin=0.7in]{geometry}
\usepackage{amsmath}
\usepackage{graphicx}
\usepackage{capt-of}
\usepackage{tcolorbox}
\usepackage{lipsum}
\usepackage{graphicx}
\usepackage{float}
\usepackage{listings}
\usepackage{hyperref} 
\usepackage{xcolor} % For custom colors
\lstset{
	language=Python,                % Choose the language (e.g., Python, C, R)
	basicstyle=\ttfamily\small, % Font size and type
	keywordstyle=\color{blue},  % Keywords color
	commentstyle=\color{gray},  % Comments color
	stringstyle=\color{red},    % String color
	numbers=left,               % Line numbers
	numberstyle=\tiny\color{gray}, % Line number style
	stepnumber=1,               % Numbering step
	breaklines=true,            % Auto line break
	backgroundcolor=\color{black!5}, % Light gray background
	frame=single,               % Frame around the code
}
\usepackage{float}
\usepackage[]{amsthm} %lets us use \begin{proof}
\usepackage[]{amssymb} %gives us the character \varnothing

	\title{Homework 3, STAT 5261}
	\author{Zongyi Liu}
	\date{Wed, Sept 24, 2025}
	\begin{document}
		\maketitle
		
		\section{Question 1}
		\subsection{Problem 1}
		Suppose that there are two risky assets, A and B, with expected returns equal to 2.3\% and 4.5\%, respectively. Suppose that the standard deviations of the returns are 6\% and 11\% and that the returns on the assets have a correlation of 0.17.
		
		\begin{itemize}
			\item (a) What portfolio of A and B achieves a 3 \% rate of expected return?
		    \item (b) What portfolios of A and B achieve a 5.5\% standard deviation of return? Among these, which has the largest expected return?
		\end{itemize}
	
	\textbf{Answer}
	\vspace*{0.1\textheight}
		
		\subsection{Problem 2}
		Suppose there are two risky assets, C and D, the tangency portfolio is 65\% C and 35\% D, and the expected return and standard deviation of the return on the tangency portfolio are 5\% and 7\%, respectively. Suppose also that the risk-free rate of return is 1.5\%. If you want the standard deviation of your return to be 5\%, what proportions of your capital should be in the risk-free asset, asset C, and asset D?
		
	
	
	\textbf{Answer}
		
		
		\clearpage
		
	\section{Question 2}
	Let $f(x,y)$ be a continuous and differentiable function of $x$ and $y$. 
	The function $f$ is said to be homogeneous of degree one if
	\[
	f(cx,cy) = c f(x,y).
	\]
	
	Euler’s theorem states that if $f$ is homogeneous of degree one, then
	\[
	f(x,y) = x \frac{\partial}{\partial x} f(x,y) + y \frac{\partial}{\partial y} f(x,y).
	\]
	
	Let 
	\[
	\mu_P(x,y) = x\mu_A + y\mu_B, 
	\quad 
	\sigma_P(x,y) = \sqrt{x^2\sigma_A^2 + y^2\sigma_B^2 + 2xy\sigma_{AB}}.
	\]
	
	\begin{enumerate}
		\item Show that $\mu_P(x,y)$ and $\sigma_P(x,y)$ are homogeneous of degree 1 (for $\sigma_P(x,y)$ assume $c \geq 0$).
		
		\item In a portfolio where the portion $x$ is invested in asset $A$ and the portion $y$ is invested in asset $B$, the partial derivatives
		\[
		\frac{\partial}{\partial x}\sigma_P(x,y) 
		\quad \text{and} \quad 
		\frac{\partial}{\partial y}\sigma_P(x,y),
		\]
		are called the marginal contributions to risk of $A$ and $B$, respectively. The contributions to risk of assets $A$ and $B$ are given by
		\[
		x \frac{\partial}{\partial x}\sigma_P(x,y) 
		\quad \text{and} \quad 
		y \frac{\partial}{\partial y}\sigma_P(x,y),
		\]
		respectively. Find the expression of the marginal risks and the contribution to risk of assets $A$ and $B$.
	\end{enumerate}


\textbf{Answer}

	\clearpage
	
	\section{Question 3}
	
	The annual estimates of the parameters for Boeing (B) and Microsoft (M) stocks are given below:
	\[
	\mu_B = 0.1492,\qquad 
	\mu_M = 0.3308,\qquad
	\sigma_B^2 = 0.0695,\qquad
	\sigma_M^2 = 0.1369,\qquad
	\rho_{BM} = -0.0083 .
	\]
	Assume a risk--free rate of $6\%$ per year for the T--bill (risk--free rate).
	
	\begin{enumerate}
		\item Use the Lagrange multiplier method to derive the minimum variance portfolio.
		\item Find the tangency portfolio and compute its mean and risk.
		\item Suppose you desire a portfolio with an expected return of $20\%$. 
		What should be the weights of this portfolio if you only use risky assets? 
		What is its risk equal to?
		\item Suppose you desire a portfolio with a risk of $20\%$. 
		What should be the weights of this portfolio if you use risky and risk--free assets? 
		What is its expected return equal to?
	\end{enumerate}


   \textbf{Answer}
   
	
	
	
	\end{document}

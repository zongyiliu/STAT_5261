\documentclass[letterpaper]{article} 
\usepackage[utf8]{inputenc}
\usepackage[T1]{fontenc}
\usepackage{amsmath}
\usepackage{amsfonts}
\usepackage{amssymb}
\usepackage{array}
\usepackage{booktabs}
\usepackage{hyperref}
\usepackage{physics}
\usepackage[version=4]{mhchem}
\usepackage{stmaryrd}
\usepackage[dvipsnames]{xcolor}
\colorlet{LightRubineRed}{RubineRed!70}
\colorlet{Mycolor1}{green!10!orange}
\definecolor{Mycolor2}{HTML}{00F9DE}
\usepackage{graphicx}
\usepackage{amsmath}
\usepackage{graphicx}
\usepackage{capt-of}
\usepackage{lipsum}
\usepackage{fancyvrb}
\usepackage{tabularx}
\usepackage{listings}
\usepackage[export]{adjustbox}
\graphicspath{ {./images/} }
\usepackage[utf8]{inputenc}
\usepackage[english]{babel}
\usepackage{float}
\usepackage{lipsum}
\usepackage{graphicx}
\usepackage{float}
\usepackage[margin=0.7in]{geometry}
\usepackage{amsmath}
\usepackage{graphicx}
\usepackage{capt-of}
\usepackage{tcolorbox}
\usepackage{lipsum}
\usepackage{graphicx}
\usepackage{float}
\usepackage{listings}
\usepackage{hyperref} 
\newcommand{\Var}{\mathrm{Var}}
\newcommand{\Cov}{\mathrm{Cov}}
\newcommand{\E}{\mathbb{E}}
\usepackage[normalem]{ulem}
\usepackage{xcolor} % For custom colors
\lstset{
	language=Python,                % Choose the language (e.g., Python, C, R)
	basicstyle=\ttfamily\small, % Font size and type
	keywordstyle=\color{blue},  % Keywords color
	commentstyle=\color{gray},  % Comments color
	stringstyle=\color{red},    % String color
	numbers=left,               % Line numbers
	numberstyle=\tiny\color{gray}, % Line number style
	stepnumber=1,               % Numbering step
	breaklines=true,            % Auto line break
	backgroundcolor=\color{black!5}, % Light gray background
	frame=single,               % Frame around the code
}
\usepackage{float}
\usepackage[]{amsthm} %lets us use \begin{proof}
	\usepackage[]{amssymb} %gives us the character \varnothing
	
	\title{Homework 8, STAT 5261}
	\author{Zongyi Liu}
	\date{Wed, Nov 12, 2025}
	
	\begin{document}
		\maketitle
		
		
		\section{Question 1}
	Let $X$ be a random variable and let $\operatorname{VaR}(\alpha)(X)$ be the value at risk corresponding to $X$. Show that
	
	\begin{enumerate}
		\item [(a)] Translation Invariance:

\begin{equation*}
\operatorname{VaR}(\alpha)(X+a)=\operatorname{VaR}(\alpha)(X)+a, \quad \forall a \in \mathbb{R} \tag{1}
\end{equation*}

\item [(b)] Positive Homogeneity:

\begin{equation*}
\operatorname{VaR}(\alpha)(\lambda X)=\lambda \operatorname{VaR}(\alpha)(X), \quad \forall \lambda \geq 0 \tag{2}
\end{equation*}

\end{enumerate}

		
			\textbf{Answer}
			
			
			
			\clearpage
			
			
			\section{Question 2}
			
			Suppose the return $R$ on a stock satisfies
	
$$
R=\mu+\lambda Y
$$

where $\mu$ and $\lambda$ are fixed and $Y$ has a $t$-distribution with $\nu$ degrees of freedom.

\begin{enumerate}
	\item [(a)] If you hold a position of size $S_{0}$ in this stock, show that for one day

$$
\operatorname{VaR}(\alpha)=-S_{0}\left(\mu+\lambda t_{\alpha, \nu}\right)
$$

where $t_{\alpha, \nu}$ is the $\alpha$ th quantile of a $t$-distribution with $\nu$ degrees of freedom. Hint: Recall that $\operatorname{Pr}(L>\operatorname{VaR}(\alpha))=\alpha$ and $L=-S_{0} R$.

\item [(b)] If $S_{0}=100,000, \mu=0.4$, and $\lambda=0.01$, compute $\operatorname{VaR}(0.05)$ when $\nu=10$.

\end{enumerate}

\textbf{Answer}



\clearpage


\section{Question 3}

Suppose the daily returns ( $R_{A}, R_{B}$ ) on Stocks A and B have a bivariate normal distribution with

$$
\boldsymbol{\mu}=\binom{0.0002}{0.0003}, \quad \Sigma=\left(\begin{array}{ll}
	0.0003 & 0.0002 \\
	0.0002 & 0.0004
\end{array}\right) .
$$

This implies that

$$
R_{A} \sim N(0.0002,0.0003), \quad R_{B} \sim N(0.0003,0.0004),
$$

and for any $a, b$,

$$
a R_{A}+b R_{B} \sim N\left(0.0002 a+0.0003 b, 0.0003 a^{2}+0.0004 b^{2}+0.0004 a b\right)
$$

\begin{enumerate}
	\item [(a)] Suppose that you hold a $\$ 1000$ position in Stock A (i.e. $S_{0}=1000$). Compute $\mathrm{VaR}_{A}(0.05)$.
	\item [(b)]  Suppose that you hold a $\$ 1000$ position in Stock B (i.e. $S_{0}=1000$). Compute $\mathrm{VaR}_{B}(0.05)$.
	\item [(c)]  What is $\operatorname{VaR}(0.05)$ of a portfolio holding $\$ 500$ in Stock A and $\$ 500$ in Stock B?
\end{enumerate}

\textbf{Answer}



\clearpage

\section{Question 4}
Suppose the distribution of $R$ has a pdf $f$. Show that

$$
\mathrm{ES}(\alpha)=-S_{0} \frac{\int_{-\infty}^{q_{\alpha}} r f(r) d r}{\alpha}
$$

where $q_{\alpha}$ is the $\alpha$ th quantile of the distribution of $R$.

\textbf{Answer}



\clearpage

\section{Question 5}

Assume $R \sim N\left(\mu, \sigma^{2}\right)$. Show that

$$
\operatorname{ES}(\alpha)=-S_{0}\left[\mu-\sigma \frac{1}{\alpha \sqrt{2 \pi}} e^{-z_{\alpha}^{2} / 2}\right]
$$

where $z_{\alpha}$ is the $\alpha$ th quantile of the standard normal distribution $N(0,1)$.

\textbf{Answer}



\clearpage

	\end{document}
	

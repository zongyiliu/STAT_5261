\documentclass[letterpaper]{article} 
\usepackage[utf8]{inputenc}
\usepackage[T1]{fontenc}
\usepackage{amsmath}
\usepackage{amsfonts}
\usepackage{amssymb}
\usepackage{array}
\usepackage{booktabs}
\usepackage{hyperref}
\usepackage[version=4]{mhchem}
\usepackage{stmaryrd}
\usepackage[dvipsnames]{xcolor}
\colorlet{LightRubineRed}{RubineRed!70}
\colorlet{Mycolor1}{green!10!orange}
\definecolor{Mycolor2}{HTML}{00F9DE}
\usepackage{graphicx}
\usepackage{amsmath}
\usepackage{graphicx}
\usepackage{capt-of}
\usepackage{lipsum}
\usepackage{fancyvrb}
\usepackage{tabularx}
\usepackage{listings}
\usepackage[export]{adjustbox}
\graphicspath{ {./images/} }
\usepackage[utf8]{inputenc}
\usepackage[english]{babel}
\usepackage{float}
\usepackage{lipsum}
\usepackage{graphicx}
\usepackage{float}
\usepackage[margin=0.7in]{geometry}
\usepackage{amsmath}
\usepackage{graphicx}
\usepackage{capt-of}
\usepackage{tcolorbox}
\usepackage{lipsum}
\usepackage{graphicx}
\usepackage{float}
\usepackage{listings}
\usepackage{hyperref} 
\usepackage{xcolor} % For custom colors
\lstset{
	language=Python,                % Choose the language (e.g., Python, C, R)
	basicstyle=\ttfamily\small, % Font size and type
	keywordstyle=\color{blue},  % Keywords color
	commentstyle=\color{gray},  % Comments color
	stringstyle=\color{red},    % String color
	numbers=left,               % Line numbers
	numberstyle=\tiny\color{gray}, % Line number style
	stepnumber=1,               % Numbering step
	breaklines=true,            % Auto line break
	backgroundcolor=\color{black!5}, % Light gray background
	frame=single,               % Frame around the code
}
\usepackage{float}
\usepackage[]{amsthm} %lets us use \begin{proof}
	\usepackage[]{amssymb} %gives us the character \varnothing
	
	\title{Homework 2, STAT 5261}
	\author{Zongyi Liu}
	\date{Fri, Sep 19, 2025}
	\begin{document}
		\maketitle
		
		\section{Question 0}
		\subsection{Part 1}
		
		Use the plot to estimate graphically the yield to maturity. Does this estimate agree with that from spline interpolation?
		
		As an alternative to interpolation, the yield to maturity can be found using a nonlinear root finder (equation solver) such as \texttt{uniroot()}, which is illustrated here:
		
		\texttt{uniroot(function(r) r\^{}2 - .5, c(0.7, 0.8))}
		
		
		\textbf{Answer}
		
		
		
		\subsection{Part 2}
		
		What does the code 	\texttt{uniroot(function(r) r\^{}2 - 0.5, c(0.7, 0.8))} do?
		
		\textbf{Answer}
		
		
			\subsection{Part 3}
		
		Use \texttt{uniroot()} to find the yield to maturity of the 30-year par
		\$1,000 bond with coupon payments of \$40 that is selling at \$1,200.
			
			
			\textbf{Answer}

			
				\subsection{Part 4}
				
		Find the yield to maturity of a par \$10,000 bond selling at \$9,800
		with semiannual coupon payments equal to \$280 and maturing in 8 years.
		
		
		\textbf{Answer}
		

			
		\subsection{Part 5}
		
		Use \texttt{uniroot()} to find the yield to maturity of the 20-year par
		\$1,000 bond with semiannual coupon payments of \$35 that is selling at \$1,050.
		
		\textbf{Answer}
		
		
\clearpage

		\section{Question 1}

Suppose that the forward rate is r(t) = 0.028+0.00042t.

\begin{itemize}
	\item (a) What is the yield to maturity of a bond maturing in 20 years?
	\item (b) What is the price of a par \$1,000 zero-coupon bond maturing in
	15 years?
\end{itemize}


\textbf{Answer}

\vspace*{0.1\textheight}

	\section{Question 3}
	
	A coupon bond has a coupon rate of 3\% and a current yield of 2.8\%. 

	\begin{itemize}
		\item (a) Is the bond selling above or below par? Why or why not?
	\item 	(b) Is the yield to maturity above or below 2.8 \%? Why or why not?
\end{itemize}
	
	\textbf{Answer}
	
\vspace*{0.1\textheight}
\section{Question 4}

Suppose that the forward rate is $r(t) = 0.032 + 0.001t + 0.0002t^2$.
	\begin{itemize}
	\item (a) What is the 5-year continuously compounded spot rate?
\item 	(b) What is the price of a zero-coupon bond that matures in 5 years?
\end{itemize}

\textbf{Answer}


\clearpage
\section{Question 7}
One year ago a par \$1,000 20-year coupon bond with semiannual coupon payments was issued. The annual interest rate (that is, the coupon rate) at that time was 8.5\%. Now, a year later, the annual interest rate is 7.6\%.

	\begin{itemize}
	\item (a) What are the coupon payments?
	\item (b) What is the bond worth now? Assume that the second coupon payment was just received, so the bondholder receives an additional 38 coupon payments, the next one in 6 months.
	\item (c) What would the bond be worth if instead the second payment were just about to be received?

\end{itemize}

\textbf{Answer}

\vspace*{0.1\textheight}

\section{Question 15}
	Suppose that a bond pays a cash flow $C_i$ at time $T_i$ for $i=1,\ldots,N$.
	Then the net present value (NPV) of cash flow $C_i$ is
	\[
	\mathrm{NPV}_i \;=\; C_i \exp(-T_i\,y_{T_i}).
	\]
	
	Define the weights
	\[
	\omega_i \;=\; \frac{\mathrm{NPV}_i}{\sum_{j=1}^{N}\mathrm{NPV}_j}
	\]
	and define the duration of the bond to be
	\[
	\mathrm{DUR} \;=\; \sum_{i=1}^{N}\omega_i T_i,
	\]
	which is the weighted average of the times of the cash flows. Show that
	\[
	\left.\frac{d}{d\delta}\sum_{i=1}^{N} C_i \exp\!\big\{-T_i\,(y_{T_i}+\delta)\big\}\right|_{\delta=0}
	= -\,\mathrm{DUR}\,\sum_{i=1}^{N} C_i \exp\{-T_i\,y_{T_i}\},
	\]
	and use this result to verify Eq.~(3.31).
	
	
	\textbf{Answer}
	

\clearpage

\section{Question 21}
	A par \$1,000 bond matures in 4years and pays semiannual coupon payments of \$25. The price of the bond is \$1,015. What is the semiannual yield to maturity of this bond?
	
	\textbf{Answer}
	
	
\end{document}

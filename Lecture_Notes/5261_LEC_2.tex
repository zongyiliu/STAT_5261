\documentclass[letterpaper]{article}
\usepackage[utf8]{inputenc}
\usepackage[T1]{fontenc}
\usepackage{amsmath}
\usepackage{amsfonts}
\usepackage{amssymb}
\usepackage{hyperref}
\usepackage[version=4]{mhchem}
\usepackage{stmaryrd}
\usepackage[dvipsnames]{xcolor}
\colorlet{LightRubineRed}{RubineRed!70}
\colorlet{Mycolor1}{green!10!orange}
\definecolor{Mycolor2}{HTML}{00F9DE}
\usepackage{graphicx}
\usepackage{amsmath}
\usepackage{graphicx}
\usepackage{capt-of}
\usepackage{lipsum}
\usepackage{fancyvrb}
\usepackage{tabularx}
\usepackage{listings}
\usepackage[export]{adjustbox}
\graphicspath{ {./images/} }
\usepackage[utf8]{inputenc}
\usepackage[english]{babel}
\usepackage{float}
\usepackage{lipsum}
\usepackage{graphicx}
\usepackage{float}
\usepackage[margin=0.7in]{geometry}
\usepackage{amsmath}
\usepackage{graphicx}
\usepackage{capt-of}
\usepackage{tcolorbox}
\usepackage{lipsum}
\usepackage{graphicx}
\usepackage{float}
\usepackage{listings}
\usepackage{hyperref} 
\usepackage{xcolor} % For custom colors
\lstset{
	language=Python,            % Choose the language  (e.g., Python, C, R) 
	basicstyle=\ttfamily\small, % Font size and type
	keywordstyle=\color{blue}, % Keywords color
	commentstyle=\color{gray}, % Comments color
	stringstyle=\color{red}, % String color
	numbers=left,           % Line numbers
	numberstyle=\tiny\color{gray}, % Line number style
	stepnumber=1,           % Numbering step
	breaklines=true,        % Auto line break
	backgroundcolor=\color{black!5}, % Light gray background
	frame=single,           % Frame around the code
}
\usepackage{float}
\usepackage[]{amsthm} %lets us use \begin{proof}
	\usepackage[]{amssymb} %gives us the character \varnothing
	\usepackage{ctex} 
	
	\title{Lecture 2, MATH 5261 \\
		\small{Fixed Income Securities\\
			固定收入證券
		}
	}
	
	\author{Zongyi Liu}
	\date{Fri, Sept 12, 2025}
	
	\begin{document}
		\maketitle
		
		\tableofcontents
		
		\section{關鍵詞}
		\begin{itemize}
			\item 請手寫:
		\end{itemize}
		
		
		\section{債券}
		\subsection{引入}
		\begin{itemize}
			\item 當你持有一家公司的股票時, 你擁有該公司部分的所有權, 這表示你同時分享其獲利與虧損 (但不承擔債務責任).  然而, 沒有任何報酬是保證的.  
			\item 當你購買債券時, 實際上是向公司提供貸款.  該公司有義務 (除非發生違約) 在貸款契約所指定的若干年後 (稱為到期日, maturity) 償還本金 (稱為票面價值, par), 並定期支付稱為息票 (coupon) 的利息.  
			\item 息票與股票股利類似, 但不同之處在於: 息票是契約上明定的支付, 而股票股利則取決於公司管理層的決策.  
			\item 投資者將因此獲得一筆固定的收入流.  
			\item 因此債券被稱為固定收益證券 (fixed-income securities).  
		\end{itemize}
		
		\subsection{零息債券}
		\begin{itemize}
			\item 零息債券 (zero-coupon bonds) 在到期前不支付任何利息或本金.  
			\item 票面價值 (par value 或 face value) 是債券在到期時支付給持有人的金額.  
			\item 若債券的殖利率為正, 零息債券的售價將低於票面價值.  
			\item 任何以低於票面價值出售的債券稱為折價債券 (discount bond).  
			\item 零息債券亦稱為純折價債券 (pure discount bond).  
		\end{itemize}
		
		
		\subsection{發行和使用}
		\begin{itemize}
			\item 借款人 (亦稱發行人, issuer) 可以是企業或政府 (例如美國財政部, 其發行的債券稱為「國庫券」 (Treasuries) ).  
			\item 債券表面上看似無風險, 但實際上並非如此.  
			\item 發行人 (借款人) 可能無法履行其支付義務, 這種情況稱為信用風險 (credit risk).  
			\item 為補償該額外風險, 貸款人要求高於一般市場利率的額外報酬, 稱為信用利差 (credit spread).  
		\end{itemize}
		
		
		\subsection{現價}
		\begin{itemize}
			\item 債券 (以及任何金融商品) 的定價皆基於一個簡單的原理——現值評價 (present valuation).  
			\item 若未來有一筆承諾支付的金額, 那麼我今天應該願意支付多少? 
			\item 回答此問題的一種方式是考慮替代方案: 在當前的市場利率下, 今天需要投資多少金額, 經過複利後, 才能在相同的未來日期獲得相同的支付金額? 
			
			\item 我們已經看到, 總報酬 (gross return) 等於期末價值除以期初價值, 且總報酬等於 \(1+\) 淨報酬 (net return).  
			\item 由於今日的 \(x\) 元在以年化利率 \(r\) 投資一年後將變為 \(x (1+r) \), 因此, 一年後可收取 1, 000 元的現值 (present value) 為: 
		\end{itemize}
		
		
		$$
		\frac{1000}{1+r}=1000 \times (1+r) ^{-1}
		$$
		
		\begin{itemize}
			\item 用於現值評價的利率稱為折現率 (discount rate).  
			\item 折現因子 \( (1 + r) ^{-1}\) 一般為時間的函數 (在上述例子中默認為一年期).  
			\item 折現函數取決於當前距離到期的時間 \(T\), 我們將其記為 \(D (T) \).  
			\item 請注意, 折現率可為無風險利率 (risk-free rate), 或為無風險利率加上先前討論的信用利差 (credit spread).  
		\end{itemize}
		
		\section{零息債券}
		\begin{itemize}
			\item 我們剛提到, 一年後可收取 1, 000 元的現值 (present value) 為 \(\dfrac{1000}{1 + r}\).  
			\item 一般公式: 零息債券 (zero-coupon bond) 的價格可表示為: 
		\end{itemize}
		
		$$
		\operatorname{PRICE}=\frac{\operatorname{PAR}}{ (1+r) ^{T}}=\operatorname{PAR} (1+r) ^{-T}
		$$
		
		若假設 \( T \) 為距到期的年數 (time to maturity in years).
		
		
		\begin{itemize}
			\item 使零息債券折現後的票面價值 (PAR) 等於其當前市場價格的年利率 \( r \), 稱為該債券的到期收益率 (yield to maturity, 簡稱 YTM).  
		\end{itemize}
		
		\subsection{零息債券的現價}
		\begin{itemize}
			\item 對於一張期限為 10 年, 票面價值為 \$1000, 利率 \( r = 4\% \) 的零息債券, 其 (合理) 市場價格 (即現值, PV) 為: 
		\end{itemize}
		
		
		$$
		\frac{1000}{ (1+0.04) ^{10}}
		$$
		
		如果利息是年複利如下: 
		
		\begin{itemize}
			\item 當利率為 \( r = 6\% \) 且按年複利計算時, 票面價值為 \$1000, 期限為 20 年的零息債券, 其價格為: 
			
		\end{itemize}
		
		$$
		\frac{1000}{ (1+r) ^{20}}=\frac{1000}{1.06^{20}}=311.80
		$$
		
		\subsection{信用利差}
		\begin{itemize}
			\item 以上 (以及以下的內容) 均假設不存在違約風險.  
			\item 但若同樣的 20 年期零息債券以 \$290 的價格出售, 則其信用利差 (credit spread) \( c \) 為多少? 
		\end{itemize}
		
		
		$$
		\frac{1000}{ (1+ (r+c) ) ^{20}}=290
		$$
		
		\begin{itemize}
			\item 有大量研究 (包含數學與統計研究) 試圖將信用利差 \(c\) 與發行人的違約機率及回收率建立關聯.  
		\end{itemize}
		
		\section{複利債券}
		\subsection{半年化複利債券}
		\begin{itemize}
			\item 若假設利率 \( r \) 為年利率並採半年複利計算, 則其價格為: 
		\end{itemize}
		
		$$
		\operatorname{PRICE}=\operatorname{PAR} (1+r / 2) ^{-2 T}
		$$
		
		\begin{itemize}
			\item 對於一張期限為 \( T \) 年, 每年複利 \( n \) 次的零息債券, 其價格為: 
		\end{itemize}
		
		$$
		\text { Price }=\operatorname{PAR} (1+r / n) ^{-n T}
		$$
		
		\begin{itemize}
			\item 若利率為 \( r = 6\% \), 且每半年複利一次, 則該 20 年期債券的現值 (present value) 為: 
		\end{itemize}
		
		
		$$
		\frac{1000}{ (1+r / 2) ^{40}}=\frac{1000}{1.03^{40}}=306.56
		$$
		
		
		\begin{itemize}
			\item 若利率 \( r = 6\% \) 採連續複利 (continuously compounded), 則該 20 年期債券的現值 (present value) 為: 
		\end{itemize}
		
		$$
		\frac{1000}{e^{20 r}}=\frac{1000}{e^{20 (0.06) }}=301.19
		$$
		
		\subsection{零息債券價格對利率上升的敏感性}
		\begin{itemize}
			\item 假設你剛以 \$306.56 購入一張票面價值為 \$1000, 期限為 20 年的零息債券.  
			\item 你假設半年複利, 計算得到的利率為 \(6\%\).  
			\item 六個月後, 市場利率上升至 \(7\%\), 此時該債券的現值 (present value) 為: 
		\end{itemize}
		
		$$
		\frac{1000}{1.035^{39}}=261.41
		$$
		
		\begin{itemize}
			\item 你的投資價值下跌了 \(306.56 - 261.41 = 45.15\).  
			\item 若你繼續持有該債券 19.5 年, 最終仍可獲得 \$1000; 然而, 若你現在出售, 將損失 \$45.15, 其報酬率為: 
		\end{itemize}
		
		$$
		\frac{-45.15}{306.56}=-14.73 \%
		$$
		
		半年期報酬率為 \(-14.73\%\), 折合年化報酬率約為 \(-29.46\%\).  
		
		\begin{itemize}
			\item 當利率上升時, 債券的價值會下降.
		\end{itemize}
		
		\subsection{零息債券價格對利率下降的敏感性}
		\begin{itemize}
			\item 假設採用半年複利.  
			\item 你剛以 \$306.56 購入該零息債券.  
			\item 六個月後, 市場利率下降至 \(5\%\), 此時該債券的現值 (present value) 為: 
		\end{itemize}
		
		$$
		\frac{1000}{1.025^{39}}=381.74
		$$
		
		\begin{itemize}
			\item 你的投資價值上升了 \(381.74 - 306.56 = 75.18\).  
			\item 若你繼續持有該債券 19.5 年, 最終仍可獲得 \$1000; 若你現在出售, 將實現 \$75.18 的利潤, 其報酬率為: 
		\end{itemize}
		
		$$
		\frac{75.18}{306.56}=24.5 \%
		$$
		
		半年期報酬率為 \(24.5\%\), 折合年化報酬率 (依債券慣例) 約為 \(49\%\).  
		
		\begin{itemize}
			\item 當利率下降時, 債券的價值會上升.  
		\end{itemize}
		
		\section{附息債券}
		
		\subsection{引入}
		這裡討論的是附息債券  (Coupon Bonds), 大多數債券會定期支付固定利息 (稱為息票, coupon).  
		
		\begin{itemize}
			\item 考慮一張期限為 20年, 票面價值為 \$1000, 年息票率 \(6\%\) 的債券, 並採半年付息.  
			\item 這表示每期息票支付金額為 \(6\% / 2 \times \$1000 = \$30\).  
			\item 債券持有人將收到共 40 期, 每期 \$30 的息票支付.  
			\item 並在 20 年後收回票面價值 \$1000.  
			\item 息票與票面價值的比率稱為息票率 (coupon rate) 或票面收益率 (coupon yield).  
			\item 息票與債券當前市場價格的比率稱為當期收益率 (current yield).  
		\end{itemize}
		
		
		\subsection{附息債券與多張零息債券的等價關係}
		
		附息債券與多張零息債券的等價關係 (亦稱為「複製原理」 (replication) ) 是金融學中的一項基本原理, 它允許我們利用其他金融工具的價格來推導某一金融資產的定價.  
		
		以三張零息債券複製一張面值為 \$1000, 年息票率 10\%, 期限三年的債券如下: 
		
		
		\begin{figure}[h]
			\begin{center}
				\includegraphics[width=0.6\textwidth]{2025_10_20_1725dd1e33e7a5fec5b4g-15}
				\caption{三種債券的收益結果}
			\end{center}
		\end{figure}
		
		\subsection{現價}
		\begin{itemize}
			\item 每筆現金流會根據其發生的期數進行折現.  
			\item 假設每半年利率為常數 \( r \).  
			\item 第一筆金額為 \( C \) 的息票, 其現值為 \(\dfrac{C}{1 + r}\) ; 第二筆息票的現值為 \(\dfrac{C}{ (1 + r) ^{2}}\), 以此類推.  
			\item 其一般公式為: 
		\end{itemize}
		
		$$
		\text {債券價格}=\sum_{t=1}^{2 T} \frac{C}{ (1+r) ^{t}}+\frac{P A R}{ (1+r) ^{2 T}}
		$$
		
		其中 \(\mathrm{PAR}\) 表示票面價值 (par value), \(\mathrm{T}\) 表示到期時間 (maturity, 以年為單位).
		
		教科書同時給出了一個相當特殊的債券定價表達式: 
		
		$$
		\text { 債券價格 }=\frac{C}{r}+\left\{\mathrm{PAR}-\frac{C}{r}\right\} (1+r) ^{-2 T}
		$$
		
		此公式的推導基於以下關係 (當 \( a \neq 1 \) 時): 
		
		\[
		1 + a + a^{2} + \ldots + a^{n} = \frac{1 - a^{n+1}}{1 - a}
		\]
		
		其中 \( a = \frac{1}{1 + r} \).  
		
		此公式在實務中並不常用, 因為金融從業人員通常會使用專門的軟體進行計算.    
		但更重要的是——這裡有一條在金融業生存的建議:  一個人人皆知但不夠優雅的公式, 永遠比一個漂亮但鮮為人知的公式更實用.  原因之一是, 你得花額外的時間向所有人證明你的公式是正確的.  
		
		
		\subsection{息票利率的折現}
		
		\begin{itemize}
			\item 假設以息票利率 (coupon rate) 進行折現  (discount).  
			\item 當以年利率 \(6\%\)  (即半年利率 \(3\%\) ) 進行折現時, 所有支付的現值總和等於 \$1000: 
		\end{itemize}
		
		$$
		\sum_{t=1}^{40} \frac{30}{1.03^{t}}+\frac{1000}{1.03^{40}}=1000
		$$
		
		\begin{itemize}
			\item 換句話說, 若當前市場利率 (用於計算現值) 等於息票收益率, 且採半年複利折現, 則債券的合理價值必定等於票面金額.  
			\item 6個月後, 若折現率保持不變, 則該債券 (包含現已到期的第一期息票支付) 的價值為: 
		\end{itemize}
		
		
		$$
		\sum_{t=0}^{39} \frac{30}{1.03^{t}}+\frac{1000}{1.03^{39}}=1030
		$$
		
		\begin{itemize}
			\item 在上述表達式中, 雖然仍包含 40 期息票, 但第一次的時間為零, 因為第一期息票現已到期.  
			\item 我們可以輕易驗證, 上述表達式等於: 
		\end{itemize}
		
		
		$$
		\left (\sum_{t=1}^{40} \frac{30}{1.03^{t}}+\frac{1000}{1.03^{40}}\right) \times 1.03=1030
		$$
		
		\begin{itemize}
			\item 稍微想一想這些結果的一致性: 在債券發行時, \$30 的息票於 6 個月後到期, 與其他所有支付一樣被折現; 但現在它已到期, 因此不再折現.  
			\item 息票剛支付完的瞬間, 尚餘 39 期息票未到期; 在相同的折現率下, 債券的現值又回到 \$1{, }000.  
		\end{itemize}
		
		\subsection{現值對利率上升的敏感}
		
		\begin{itemize}
			\item 假設在債券發行6個月後, 市場利率 (或折現率) 上升至 \(7\%\).  
			\item 則六個月後, 該債券的價值為: 
		\end{itemize}
		
		
		$$
		\sum_{t=0}^{39} \frac{30}{1.035^{t}}+\frac{1000}{1.035^{39}}=924.48
		$$
		
		年化收益為: 
		
		$$
		2\left (\frac{924.48-1000}{1000}\right) =-15.1 \%
		$$
		
		\begin{itemize}
			\item 當利率上升時, 債券的價值會下降.  
			\item 若6個月後利率下降至 \(5\%\), 則該投資的價值為: 
		\end{itemize}
		
		$$
		\sum_{t=0}^{39} \frac{30}{1.025^{t}}+\frac{1000}{1.025^{39}}=1153.70
		$$
		
		由此導致了一個年化收益如下: 
		
		$$
		2\left (\frac{1153.70-1000}{1000}\right) =30.7 \%
		$$
		
		\begin{itemize}
			\item 當利率下降時, 債券的價值會上升.  
		\end{itemize}
		
		\section{到期收益率 (YTM) }
		
		\begin{itemize}
			\item 假設一張債券的剩餘到期時間為 \(T = 30\) 年, 並採用年複利計算.  
			\item 年息票金額為 \(C = 40\, \$\), 票面價值 (PAR) 為 \(1000\, \$\).  
			\item 假設目前市場價格為 \$1200, 比票面價值高出 \$200.   (到期時, 你只會收到投資的 \$1200 中的 \$1000.  ) 
			\item 我們已知 \(40 / 1000 = 4\%\) 為息票率 (coupon rate).  
			\item 但你投入 \$1200 只收到 \$40 的息票, 因此當期收益率為每半年 \(3.33\%\), 或年化 \(6.66\%\).  
			\item 到期收益率 (YTM) 是使債券的現值等於實際市場價格的那個折現率.  
		\end{itemize}
		
		$$
		\sum_{t=1}^{30} \frac{40}{(1+\text{YTM})^{t}} + \frac{1000}{(1+\text{YTM})^{30}} = \text{現價}
		$$
		
		在此例中, YTM 為每半年 \(0.0324\), 年化約為 \(6.48\%\).  
		
		
		\subsection{公式}
		\begin{itemize}
			\item 假設一張附息債券每半年支付一次息票金額 \(C\), 票面價值為 \(\mathrm{PAR}\), 距離到期尚有 \(T\) 年.  
			\item 設 \(r_{1}, r_{2}, \ldots, r_{2T}\) 為對應於未來 \(1/2, \, 1, \, 1.5, \, \ldots, \, T\) 年之半年期即期利率 (half-year spot rates).  
			\item 到期收益率 (YTM) 定義為使下式成立的單一期限利率 \(y\): 
		\end{itemize}
		
		$$
		\begin{aligned}
			\frac{C}{1+r_{1}} & +\frac{C}{\left (1+r_{2}\right) ^{2}}+\ldots+\frac{C}{\left (1+r_{2 T-1}\right) ^{2 T-1}}+\frac{C}{\left (1+r_{2 T}\right) ^{2 T}}+\frac{P A R}{\left (1+r_{2 T}\right) ^{2 T}} \\
			& =\frac{C}{1+y}+\frac{C}{ (1+y) ^{2}}+\ldots+\frac{C}{ (1+y) ^{2 T-1}}+\frac{C}{ (1+y) ^{2 T}}+\frac{P A R}{ (1+y) ^{2 T}}
		\end{aligned}
		$$
		
		\includegraphics[max width=0.6\textwidth, center]{2025_10_20_1725dd1e33e7a5fec5b4g-24}
		\captionof{figure}{債券價格與到期收益率 (YTM) 的關係如下圖所示.  紅色的水平線代表債券價格 \$1, 200; 價格–收益率曲線與此線相交於 0.0324, 如紅色垂直線所示.  因此, 0.0324 即為該債券的每半年到期收益率.  }
		
		
		若價格 (price) 大於票面價值 (par), 則有
		\[
		\frac{1}{\text{price}} < \frac{1}{\text{par}}, 
		\]
		因此, 當期收益率 (current yield) 小於息票率 (coupon rate).  
		
		
		\includegraphics[width=0.6\textwidth, center]{2025_10_20_1725dd1e33e7a5fec5b4g-25}
		\captionof{figure}{par \$=1000, coupon payment \$=40, T=30\$}
		
		
		一項關鍵特性, 如前圖所示:   
		\[
		\text{價格} > \text{票面價值  (par) } \Longrightarrow \text{息票率 (coupon rate) } > \text{當期收益率 (current yield) }
		\]
		
		反之:   
		\[
		\text{價格} < \text{票面價值} \Longrightarrow \text{息票率} < \text{當期收益率}
		\]
		
		更廣義地說, 如前圖所示, price $>$ par $\Longrightarrow$ coupon rate $>$ current yield $>$ yield to maturity.
		
		
		互利性地來說, price $<$ par $\Longrightarrow$ coupon rate $<$ current yield $<$ yield to maturity.
		
		\section{即期利率}
		我們已經看到, 對於每半年支付一次息票 (這是大多數債券的情況) 的附息債券, 其價格與其到期收益率 (YTM) \( y \) 的關係如下: 
		
		\[
		\text{bond price} = \sum_{t=1}^{2T} \frac{C}{ (1 + y) ^t} + \frac{\mathrm{PAR}}{ (1 + y) ^{2T}}
		\]
		
		對於零息債券 (zero-coupon bond), 當 \( C = 0 \) 時, 上式化簡為: 
		
		\[
		\text{PRICE} = \text{PAR} \,  (1 + y) ^{-2T}
		\]
		
		這表示, 對於無信用風險的零息債券, 其到期收益率 (YTM) 即為今日對應於期限 \( T \) 年的即期利率 (interest rate),  
		亦即「當下」 (on the spot) 的借貸利率.  
		
		
		\begin{itemize}
			\item 零息債券可用來推導「純粹」或「乾淨」的貼現率/到期收益率 (YTM), 不受中間票息支付的影響.
			\item 其到期收益率稱為即期利率 (spot rates).
			\item 然而利率會隨著時間範圍 (到期年限) 與時間本身而變化.
			\item 更準確地說, 期限為 $n$ 年的零息債券的到期收益率稱為 $n$ 年期即期利率.
			\item 在課本中, 即期利率記作 $y_{n}$.
			\item 若假設年化利率, 則 $n$ 年後支付 \$1 的現值 (或淨現值) 為 $\frac{\$1}{(1 + y_{n})^{n}}$.
		\end{itemize}
		
		
		例子: 一張期限為 4 年, 票面價值 \$100, 每年支付 \$5 息票的債券: 
		
		
		\begin{center}
			\begin{tabular}{|l|l|l|l|l|l|}
				\hline
				Years & 0 & 1 & 2 & 3 & 4 \\
				\hline
				Coupon &  & 5 & 5 & 5 & 5 \\
				\hline
				Par &  & 0 & 0 & 0 & 100 \\
				\hline
				Spot rates &  & 4\% & 4.30\% & 4.20\% & 3.59\% \\
				\hline
				PV of coupon &  & 4.81 & 4.60 & 4.42 & 91.18 \\
				\hline
				Bond price & 105.00 &  &  &  &  \\
				\hline
				Years & 0 & 1 & 2 & 3 & 4 \\
				\hline
				Coupon &  & 5 & 5 & 5 & 5 \\
				\hline
				Par &  & 0 & 0 & 0 & 100 \\
				\hline
				YTM &  & 3.63\% & 3.63\% & 3.63\% & 3.63\% \\
				\hline
				PV of coupon &  & 4.82 & 4.66 & 4.49 & 91.04 \\
				\hline
				Bond price & 105.02 &  &  &  &  \\
				\hline
			\end{tabular}
		\end{center}
		
		\subsection{YTM vs 即期利率}
		\begin{itemize}
			\item 但這僅適用於零息債券 (zero-coupon bonds).  
			\item 否則, 附息債券的到期收益率 (YTM) 僅針對特定債券: 它是那個單一的利率, 使得以該利率折現所有現金流 (息票與最終本金支付) 後, 其現值等於當前市場價格.  
			\item 相對地, 即期利率 (spot rates) 是「通用」的, 但適用於不同期限的當前時點; 市場上存在這些利率的報價.  
			\item 附息債券可視為多張零息債券的組合, 每張具有不同的到期日, 因此需以不同的即期利率進行折現.  
		\end{itemize}
		
		\subsection{給債券定價的即期利率}
		假設: 
		
		\begin{itemize}
			\item 半年期即期利率為每年 \(5\%\) ; 
			\item 一年期即期利率為每年 \(6\%\).  
		\end{itemize}
		
		我們可以將一張每半年支付 \$40 息票的債券視為由兩張零息債券所組成: 
		
		\begin{itemize}
			\item 一張期限 \(T = \tfrac{1}{2}\) 年, 票面價值為 \$40 的零息債券; 
			\item 一張期限 \(T = 1\) 年, 票面價值為 \$1040 的零息債券; 
			\item 該附息債券的價格等於這兩張零息債券價格的總和, 其表達式為: 
		\end{itemize}
		
		
		$$
		\frac{40}{1+0.025}+\frac{1040}{ (1+0.03) ^{2}}=1019 .
		$$
		
		\begin{itemize}
			\item 該附息債券的到期收益率 (YTM) 為使下式成立的 \( y \) 值: 
		\end{itemize}
		
		
		$$
		\frac{40}{1+y}+\frac{1040}{ (1+y) ^{2}}=1019 .
		$$
		
		在此中 $y=0.0299$.
		
		\subsection{案例}
		另一個例子: 已知即期利率 \(r_{1} = 8\%\), \(r_{2} = 10\%\), 那麼一張票面利率為 \(5\%\) 的兩年期債券應該定價多少? 假設按年複利計算.  
		
		例題解答: 已知即期利率 \(r_{1} = 8\%\), \(r_{2} = 10\%\), 則票面利率為 \(5\%\) 的兩年期債券價格應為 (假設年複利): 
		
		
		$$
		\text { PRICE }=\frac{50}{1+0.08}+\frac{1050}{ (1+0.10) ^{2}}=914.06
		$$
		
		\section{期限結構}
		\subsection{期限結構和產出曲線}
		\begin{itemize}
			\item 即使你的本金能保證收回, 借出資金的時間越長, 所期望的利率也應該越高.  
			\item 若將同一發行人提供的利率作為到期期限的函數繪圖, 此圖稱為收益率曲線 (term structure).  
		\end{itemize}
		
		\includegraphics[max width=0.6\textwidth, center]{2025_10_20_1725dd1e33e7a5fec5b4g-35}
		\captionof{figure}{在特例中, 若利率來自美國國庫券 (Treasuries), 則該收益率結構稱為收益率曲線 (yield curve) }
		
		\subsection{案例}
		債券的利率取決於其到期期限——也就是你必須等待多長時間才能收回投資.  在 2001年3月28日, 美國國庫券 (Treasuries) 的利率如下: 
		
		
		\begin{itemize}
			\item $4.23 \%$ for 3 -month bills
			\item $4.81 \%$ on 10 -year notes
			\item $5.46 \%$ on 30 -year bonds
		\end{itemize}
		
		在2006年3月22日如下: 
		
		\begin{itemize}
			\item $4.69 \%$ on 3 -month
			\item $4.86 \%$ on 10 -year
			\item $4.73 \%$ on 30 -year
		\end{itemize}
		
		
		\section{遠期利率}
		對於所有到期時間長達 \( n \) 年的債券, 其期限結構 (term structure) 可以用下列任一組參數描述: 
		
		\begin{itemize}
			\item 遠期利率 (forward rates) \( r_{1}, r_{2}, \ldots, r_{n} \), 其中 \( r_{k} \) 表示你現在可以鎖定的, 在未來第 \( k \) 年借出 (此處假設為一年期) 的利率.  
			\item 在此記號下, \( r_{1} \) 即為下一年的即期利率 (spot rate).  
			\item 注意, 若你將資金借出兩年, 則應該對以下兩種情況無差別:   
			第一種是兩年間均獲得利率 \( y_{2} \) ;   
			第二種是第一年獲得 \( r_{1} \), 第二年獲得 \( r_{2} \).  
			\item 這是金融學中的一項核心原理, 稱為「套利原理」 (arbitrage argument).  
			\item 由此可知, \$1 的現值可表示為 \(\dfrac{\$1}{ (1 + y_{2}) ^{2}}\), 也可表示為 \(\dfrac{\$1}{ (1 + r_{1})  (1 + r_{2}) }\).  
			\item 因此有: 
			\[
			(1 + y_{2}) ^{2} =  (1 + r_{1})  (1 + r_{2}) 
			\]
		\end{itemize}
		
		
		\begin{figure}[h]
			\begin{center}
				\caption{Forward Rates  (3.5.2) }
				\includegraphics[width=0.6\textwidth]{2025_10_20_1725dd1e33e7a5fec5b4g-38}
			\end{center}
		\end{figure}
		
		\subsection{利率的期限結構 }
		對於所有到期時間長達 \( n \) 年的債券, 其期限結構 (term structure) 可由以下任一組數據描述: 
		
		\begin{itemize}
			\item 不同到期期限 (1 年, 2 年, ⋯, \( n \) 年) 的零息債券價格, 分別記為 \( P (1), P (2), \ldots, P (n) \) ; 
			\item 不同到期期限 (1 年, 2 年, ⋯, \( n \) 年) 的即期利率 (即零息債券的收益率), 分別記為 \( y_{1}, y_{2}, \ldots, y_{n} \) ; 
			\item 遠期利率 (forward rates) \( r_{1}, r_{2}, \ldots, r_{n} \), 其中 \( r_{k} \) 表示你現在可以鎖定的, 在未來第 \( k \) 年借出資金時適用的利率.  
		\end{itemize}
		
		
		\begin{itemize}
			\item 考慮一張票面價值為 \$1000, 到期時間為 2 年的零息債券.  
			\item 一年後, 其現值 (present value, 即價格) 將等於票面金額以當時適用於下一年的利率折現所得: 
		\end{itemize}
		
		
		$$
		\frac{1000}{1+r_{2}}
		$$
		
		\begin{itemize}
			\item 上述為該債券在一年後的價值.  然而, 若要計算其今日的價值, 需再以當前利率進行折現, 因此今日價格為: 
		\end{itemize}
		
		$$
		P (2) =\frac{1000}{\left (1+r_{1}\right) \left (1+r_{2}\right) }
		$$
		
		\begin{itemize}
			\item 整體而言, 一個現價 $\$ 1000$ 在$n$分期支付的債券從現在開始是: 
		\end{itemize}
		
		$$
		P (n) =\frac{1000}{\left (1+r_{1}\right) \left (1+r_{2}\right) \ldots\left (1+r_{n}\right) }
		$$
		
		其中 \( r_{1}, r_{2}, \ldots, r_{n} \) 分別表示在第 \(1, 2, \ldots, n\) 期內適用的遠期利率 (forward interest rates).  
		
		\begin{itemize}
			\item 上述各組數據 (零息債券價格, 即期利率與遠期利率) 皆可由其他任意一組推導而得.  
			\item 我們將從遠期利率計算到期收益率 (YTM) ; 
			\item 從即期利率計算遠期利率; 
			\item 從遠期利率計算債券價格; 
			\item 並從價格計算到期收益率 (YTM).  
		\end{itemize}
		
		
		\subsection{通過遠期利率求 YTM }
		\begin{itemize}
			\item 到期收益率 \( y_{n} \) 是使下式成立的唯一利率: 
		\end{itemize}
		$$
		\frac{1}{\left (1+r_{1}\right) \left (1+r_{2}\right) \ldots\left (1+r_{n}\right) }=\frac{1}{\left (1+y_{n}\right) ^{n}}
		$$
		
		所以我們有: 
		
		$$
		y_{n}=\sqrt[n]{\left (1+r_{1}\right) \left (1+r_{2}\right) \ldots\left (1+r_{n}\right) }-1
		$$
		
		$\left (1+y_{n}\right.$ 是$\left.\left (1+r_{1}\right), \left (1+r_{2}\right), \ldots, \left (1+r_{n}\right) \right) $的幾何平均 
		
		
		\begin{itemize}
			\item $r_{1}=y_{1}$
			\item 以及
		\end{itemize}
		
		$$
		r_{n}=\frac{\left (1+y_{n}\right) ^{n}}{\left (1+y_{n-1}\right) ^{n-1}}-1
		$$
		
		\subsection{通過向前利率求價格}
		
		\begin{center}
			\begin{tabular}{cc}
				Example:  &  \\
				Year (i) & Forward interest rate $\left (r_{i}\right) $ \\
				\hline
				1 & $6 \%$ \\
				2 & $7 \%$ \\
				3 & $8 \%$ \\
			\end{tabular}
		\end{center}
		
		
		\begin{itemize}
			\item 一張期限為一年, 票面價值 \$1000 的零息債券, 其售價為: 
		\end{itemize}
		
		$$
		P (1) =\frac{1000}{1+r_{1}}=\frac{1000}{1+0.06}=943.40
		$$
		
		\begin{itemize}
			\item 一張期限為兩年的零息債券, 其售價為: 
		\end{itemize}
		
		$$
		P (2) =\frac{1000}{\left (1+r_{1}\right) \left (1+r_{2}\right) }=\frac{1000}{ (1+0.06)  (1+0.07) }=881.68
		$$
		
		\begin{itemize}
			\item 一張期限為三年的零息債券, 其售價為: 
		\end{itemize}
		
		$$
		P (3) =\frac{1000}{\left (1+r_{1}\right) \left (1+r_{2}\right) \left (1+r_{3}\right) }=\frac{1000}{ (1+0.06)  (1+0.07)  (1+0.08) }=816.37
		$$
		
		\subsection{由價格與遠期利率求得到期收益率}
		
		範例: 由價格求得到期收益率 (YTM) ——使用在範例 3.2中推導出的價格, 以及先前相同的遠期利率.  
		
		
		\begin{center}
			\begin{tabular}{cc}
				Year  (i) & Forward interest rate $\left (r_{i}\right) $ \\
				\hline
				1 & $6 \%$ \\
				2 & $7 \%$ \\
				3 & $8 \%$ \\
			\end{tabular}
		\end{center}
		
		\begin{itemize}
			\item 我們有: 
		\end{itemize}
		
		$$
		P (1) =\frac{1000}{1+r_{1}}=\frac{1000}{1+0.06}=943.40
		$$
		
		\begin{itemize}
			\item 解出到期收益率 \( y_{1} \) 得 \( 0.06 \).  
			\item 接著求 \( y_{2} \):  
		\end{itemize}
		
		$$
		P (2) =\frac{1000}{\left (1+y_{2}\right) ^{2}}=881.68
		$$
		
		因而有: 
		
		$$
		y_{2}=\sqrt{\frac{1000}{P (2) }}-1=0.0650
		$$
		
		\begin{itemize}
			\item 取得 $y_{3}$: 
		\end{itemize}
		
		$$
		P (3) =\frac{1000}{\left (1+y_{3}\right) ^{3}}=816.37
		$$
		
		因而有: 
		
		$$
		y_{3}=\sqrt[3]{\frac{1000}{P (3) }}-1=0.070
		$$
		
		\begin{itemize}
			\item 總而言之, YTM 等於: 
		\end{itemize}
		
		$$
		y_{n}=\sqrt[n]{\frac{1000}{P (n) }}-1
		$$
		
		\begin{itemize}
			\item 我們現在將先前所學應用於課本中的範例 3.3; 
			\item 我們看到, 零息債券的到期收益率 $y_{n}$ 滿足以下關係式: 
		\end{itemize}
		
		
		$$
		\frac{1}{\left (1+r_{1}\right) \left (1+r_{2}\right) \ldots\left (1+r_{n}\right) }=\frac{1}{ (1+y) ^{n}}
		$$
		
		我們有: 
		
		$$
		y_{n}=\sqrt[n]{\left (1+r_{1}\right) \left (1+r_{2}\right) \ldots\left (1+r_{n}\right) }-1
		$$
		
		\begin{itemize}
			\item $1 + y_{n}$ 是 $\left (1 + r_{1}\right), \left (1 + r_{2}\right), \ldots, \left (1 + r_{n}\right) $ 的幾何平均數; 
			\item 在本例中, 我們有: 
		\end{itemize}
		
		
		$$
		y_{1}=r_{1}=0.06, \quad y_{2}=\sqrt{\left (1+r_{1}\right) \left (1+r_{2}\right) }-1=\sqrt{ (1.06)  (1.07) }-1=0.06499
		$$
		
		以及: 
		
		$$
		y_{3}=\sqrt{\left (1+r_{1}\right) \left (1+r_{2}\right) \left (+r_{3}\right) }-1=\sqrt{ (1.06)  (1.07)  (1.08) }-1=0.0700
		$$
		
		\subsection{利率期限結構}
		總結而言, 我們可以推導出以下關於到期收益率 (YTM), 債券價格與遠期利率之間關係的方程式: 
		
		\begin{itemize}
			\item 由債券價格求得到期收益率: 
		\end{itemize}
		
		$$
		y_{n}=\sqrt[n]{\frac{1000}{P (n) }}-1
		$$
		
		\begin{itemize}
			\item 由遠期利率求得到期收益率: 
		\end{itemize}
		
		$$
		y_{n}=\sqrt[n]{\left (1+r_{1}\right) \left (1+r_{2}\right) \ldots\left (1+r_{n}\right) }-1
		$$
		
		\begin{itemize}
			\item 由到期收益率求得債券價格: 
		\end{itemize}
		
		$$
		P (n) =\frac{1000}{\left (1+y_{n}\right) ^{n}}
		$$
		
		\begin{itemize}
			\item 由零息債券的價格求得遠期利率 $r_{n}$: 
		\end{itemize}
		
		$$
		r_{n}=\frac{P (n-1) }{P (n) }-1
		$$
		
		從價格求得收益率與遠期利率 (範例 3.4) 
		
		假設一年期, 兩年期與三年期面值為 \$1, 000 的零息債券, 其價格如下表所示: 
		\begin{center}
			\begin{tabular}{cc}
				Maturity  (years) & Price \\
				\hline
				1 & $\$ 920$ \\
				2 & $\$ 830$ \\
				3 & $\$ 760$ \\
			\end{tabular}
		\end{center}
		
		\subsection{由價格求得 YTM}
		
		此處的 YTM 為: 
		
		$$
		\begin{aligned}
			& y_{1}=\frac{1000}{920}-1=0.087 \\
			& y_{2}=\left\{\frac{1000}{830}\right\}^{1 / 2}-1=0.0976 \\
			& y_{3}=\left\{\frac{1000}{760}\right\}^{1 / 3}-1=0.096
		\end{aligned}
		$$
		
		
		遠期利率: 
		
		$$
		\begin{aligned}
			& r_{1}=y_{1}=0.087 \\
			& r_{2}=\frac{\left (1+y_{2}\right) ^{2}}{\left (1+y_{1}\right) }-1=\frac{ (1.0976) ^{2}}{1.0876}-1=0.108, \text { and } \\
			& r_{3}=\frac{\left (1+y_{3}\right) ^{3}}{\left (1+y_{2}\right) ^{2}}-1=\frac{ (1.096) ^{3}}{ (1.0976) ^{2}}-1=0.092 .
		\end{aligned}
		$$
		
		\subsection{由價格求得遠期利率}
		
		遠期利率: 
		
		$$
		\begin{aligned}
			& r_{1}=\frac{1000}{920}-1=0.087 \\
			& r_{2}=\frac{920}{830}-1=0.108 \\
			& r_{3}=\frac{830}{760}-1=0.092
		\end{aligned}
		$$
		
		\section{連續複利}
		連續複利使遠期利率, 零息債券的到期收益率 (即即期利率) 與零息債券價格之間的關係更為簡化.  
		
		\begin{itemize}
			\item 由遠期利率求得價格: 
		\end{itemize}
		
		$$
		P (1) =\frac{1000}{\exp \left (r_{1}\right) }=1000 e^{-r_{1}}
		$$
		
		普遍來說: 
		
		$$
		P (n) =\frac{1000}{e^{r_{1}+r_{2}+\ldots+r_{n}}}
		$$
		
		\begin{itemize}
			\item 由價格求得遠期利率
		\end{itemize}
		
		$$
		\frac{P (n-1) }{P (n) }=\frac{e^{r_{1}+r_{2}+\ldots+r_{n}}}{e^{r_{1}+r_{2}+\ldots+r_{n-1}}}, 
		$$
		
		therefore
		
		$$
		r_{n}=\log \left\{\frac{P (n-1) }{P (n) }\right\}
		$$
		
		\begin{itemize}
			\item $n$ 年期零息債券的到期收益率需滿足以下方程式: 
		\end{itemize}
		
		
		$$
		P (n) =\frac{1000}{e^{n y_{n}}}
		$$
		
		\begin{itemize}
			\item 也就是說, 貼現函數為: 
		\end{itemize}
		
		
		$$
		e^{-n y_{n}}
		$$
		\begin{itemize}
			\item 因此, 我們可由遠期利率求得到期收益率: 
		\end{itemize}
		
		$$
		y_{n} = \frac{r_{1} + r_{2} + \ldots + r_{n}}{n}
		$$
		
		\begin{itemize}
			\item 而 $r_{1}, r_{2}, \ldots, r_{n}$ 則可由 $y_{1}, y_{2}, \ldots, y_{n}$ 求得: 
		\end{itemize}
		
		
		$$
		r_{1}=y_{1} \quad \text { and } \quad r_{n}=n y_{n}- (n-1) y_{n-1}
		$$
		對於 $n > 1$.  
		
		\subsection{由價格求得連續複利的遠期利率與到期收益率}
		
		使用與前例相同的價格: 
		
		\begin{center}
			\begin{tabular}{cc}
				Maturity  (years) & Price \\
				\hline
				1 & $\$ 920$ \\
				2 & $\$ 830$ \\
				3 & $\$ 760$ \\
			\end{tabular}
		\end{center}
		
		我們如下推導連續複利利率: 
		
		
		$$
		\begin{aligned}
			& r_{1}=\log \left\{\frac{1000}{920}\right\}=0.083 \\
			& r_{2}=\log \left\{\frac{920}{830}\right\}=0.103 \\
			& r_{3}=\log \left\{\frac{830}{760}\right\}=0.088
		\end{aligned}
		$$
		\section{連續遠期利率 }
		我們先前假設遠期利率滿足以下條件: 
		
		\begin{itemize}
			\item 在每一年內為常數; 
			\item 具有固定的起始時間, 且所有到期日皆為自該日期起的整數年; 
			\item 為使模型更貼近現實, 我們假設存在一個遠期利率函數 $r (t) $; 
			\item 我們剛剛看到貼現函數可以表示為到期收益率 (YTM) 的函數: 
		\end{itemize}
		
		$$
		e^{-n y_{n}}
		$$
		
		\begin{itemize}
			\item 因此, 其連續形式自然為: 
		\end{itemize}
		
		$$
		D (T) = e^{-\int_{0}^{T} r (t) \, dt}
		$$
		
		\begin{itemize}
			\item 當然, 債券價格仍為 $P (T) = \operatorname{PAR} \times D (T) $; 
			\item 期限為 $T$ 的零息債券之到期收益率定義為: 
		\end{itemize}
		
		$$
		y_{T} = \frac{1}{T} \int_{0}^{T} r (t) \, dt
		$$
		
		\begin{itemize}
			\item 因此, 債券的價格可表示為: 
		\end{itemize}
		
		$$
		P (T) =\operatorname{PAR} \quad e^{-T_{y_{T}}}
		$$
		
		\begin{itemize}
			\item 可以想像, 實務界人士更喜歡這個單一數值 $y_{T}$, 並且傾向使用上式來表示, 而不是: 
		\end{itemize}
		
		$$
		P (T) = \operatorname{PAR} \, e^{-\int_{0}^{T} r (t) \, dt}
		$$
		
		\section{額外幾點}
		\subsection{分段常數情況}
		\begin{itemize}
			\item 書中討論了一種特例, 即分段常數遠期利率, 但並未作太多說明; 
			\item 該範例僅在每個利率適用於完整一年度時才成立; 
			\item 否則, 積分的形式會稍微更複雜.  \\
			若 $r (t) = r_{k}$ 對於 $k - 1 \leq t \leq k$, 則: 
		\end{itemize}
		
		
		$$
		\int_{0}^{T} r (t) d t=r_{1}+r_{2}+\ldots+r_{T}
		$$
		因此, 債券價格為: 
		
		$$
		P (T) = \operatorname{PAR} \times e^{-\left (r_{1} + r_{2} + \ldots + r_{T}\right) }
		$$
		
		\subsection{線性遠期利率}
		\textbf{範例: } 假設遠期利率為
		
		$$
		r (t) = 0.03 + 0.0005t
		$$
		
		求 $r (10) $, $y_{10}$ 以及面值 \$1{, }000, 到期 10 年的零息債券價格 $P (10) $.  \\
		\textbf{解答: }
		
		\begin{itemize}
			\item 10 年後的遠期利率: $r (10) = 0.03 + 0.0005 (10) = 0.035$; 
			\item 到期收益率 (YTM): 
		\end{itemize}
		
		$$
		y_{10} = \frac{1}{10} \int_{0}^{10} (0.03 + 0.0005t) \, dt 
		= \frac{1}{10}\left (0.03 (10) + 0.0005 \frac{100}{2}\right) 
		= 0.0325
		$$
		
		\begin{itemize}
			\item 債券價格等於面值乘以貼現函數, 以剛計算的到期收益率為基礎: 
		\end{itemize}
		
		$$
		P (10) = 1000\, e^{-10 (0.0325) } = 772.5
		$$
		
		\section{久期}
		
		\subsection{附息債券久期}
		久期 (尤其是修正久期) 的直覺意涵是: 我們希望用一個數字來衡量價格對利率變動的敏感度, 即所謂的利率風險.  
		
		$$
		\% \text{ 價格變化 } = \text{ 久期 (DURATION) } \times \text{ 收益率變化 (change in yield) }
		$$
		
		其中: 
		
		$$
		\% \text{ 價格變化 } = \frac{\text{ 債券價格變動 }}{\text{ 債券價格 }}
		$$
		
		因此: 
		
		$$
		\text{ 久期 (DURATION) } = \frac{-1}{\text{ 債券價格 }} \times \frac{\text{ 債券價格變動 }}{\text{ 收益率變化 }}
		$$
		
		\begin{itemize}
			\item 注意, 收益率變化指的是利率的絕對變化; 
			\item 債券價格與價格變化以美元計; 
			\item 因此 (修正) 久期為無量綱 (unit-less).  
		\end{itemize}
		
		
		\subsection{通常久期情況}
		現在回到一般的連續利率情況 (非分段或線性利率): 
		
		\begin{itemize}
			\item 債券具有風險, 因為其價格對利率變動十分敏感, 這種問題稱為「利率風險」 (interest-rate risk) ; 
			\item 我們之前看到, 對於零息債券, 其貼現函數為: 
		\end{itemize}
		
		$$
		D (T) = \exp\!\left (-\int_{0}^{T} r (t) \, dt\right) = \exp (-T y_{T}) 
		$$
		
		\begin{itemize}
			\item 假設 $y_{T}$ 發生微小變化: 
		\end{itemize}
		
		
		$$
		\frac{d D (T) }{d y_{T}}=\frac{d}{d y_{T}} \exp \left (-T y_{T}\right) =-T \exp \left (-T_{y_{T}}\right) =-T D (T) 
		$$
		\begin{itemize}
			\item 我們先前有 $P (T) = \operatorname{PAR} \times D (T) $, 因此: 
		\end{itemize}
		
		$$
		\frac{dP (T) }{dy_{T}} = \operatorname{PAR} \times \left (-T D (T) \right) 
		$$
		
		\begin{itemize}
			\item 重新整理可得: 
		\end{itemize}
		
		$$
		\frac{dP (T) }{\operatorname{PAR} \times D (T) } = -T\, dy_{T}
		$$
		
		\begin{itemize}
			\item 換句話說: 
		\end{itemize}
		
		$$
		\frac{dP (T) }{P (T) } = -T\, dy_{T}
		$$
		
		\begin{itemize}
			\item 因此, 零息債券價格的百分比變化約等於負的到期期限乘以收益率的微小變化; 
			\item 右側的負號反映出一個眾所周知的事實: 債券價格與利率變動方向相反; 
			\item 此外, 債券價格的相對變化與 $T$ 成正比, 這體現了「長期債券比短期債券具有更高利率風險」的原理.  
		\end{itemize}
		
		
		債券具有風險, 因為其價格對利率變動十分敏感, 這種現象稱為「利率風險」 (interest-rate risk).    債券價格對收益率變動的敏感程度如何?   久期 (Duration) 定義如下, 使得: 
		
		$$
		\frac{\text{債券價格變動}}{\text{債券價格}} = \text{久期 (DUR) } \times \text{收益率變化}
		$$
		
		\begin{itemize}
			\item 那麼, 零息債券的久期為何? 
			\item 其貼現函數為: 
		\end{itemize}
		
		
		$$
		D (T) =\exp \left (-\int_{0}^{T} r (t) d t\right) =\exp \left (-T y_{T}\right) 
		$$
		\begin{itemize}
			\item 假設 $y_{T}$ 變為 $y_{T} + \delta$
		\end{itemize}
		
		$$
		\frac{d}{d y_{T}} \exp\left (-T y_{T}\right) \approx -T \exp\left (-T y_{T}\right) = -T D (T) 
		$$
		
		\begin{itemize}
			\item 因此
		\end{itemize}
		
		$$
		\frac{d P (T) }{d y_{T}} = \operatorname{PAR} \frac{d}{d y_{T}} \exp\left (-T y_{T}\right) \approx -\operatorname{PAR}\, T\, D (T) 
		$$
		
		\begin{itemize}
			\item 重新整理……
		\end{itemize}
		
		$$
		\frac{d P (T) }{\operatorname{PAR}\, D (T) } = -T\, d y_{T}
		$$
		
		\begin{itemize}
			\item 因而有:
		\end{itemize}
		
		$$
		\frac{\text{債券價格變動}}{\text{債券價格}} \approx -T \times \text{收益率變化 } \delta
		$$
		
		\begin{itemize}
			\item 右側的負號顯示債券價格與利率變動方向相反.  
		\end{itemize}
		
		
		\subsection{含息債券的久期}
		\begin{itemize}
			\item 久期 (Duration) 的最初定義稱為麥考利久期 (Macaulay Duration), 其概念建立在: 附息債券可視為由多個不同到期日的零息債券所組成.  
			\item 附息債券的麥考利久期為各現金流到期時間的加權平均數, 其中權重為各現金流現值在債券總現值中的比例.  
			\item 因此, 麥考利久期的單位為時間, 通常以「年」表示.  
			\item 注意!教材常混用兩種久期 (Macaulay 與 Modified Duration).  
			\item 若附息債券的當前價值為 $V$, 且共有 $n$ 個現金流 (包含票息與面值償還), 每個現金流的現值為 $PV_{i}$, 則該債券的麥考利久期為: 
		\end{itemize}
		
		
		$$
		\text { Macaulay Duration }=\sum_{i=1}^{n} T_{i} \frac{P V_{i}}{V}
		$$
		
		\subsection{時間遞減}
		\begin{itemize}
			\item 時間遞減 (Time Decay) 又稱「回歸面值效應」 (Pull to Par) 
			\item 利率變動並非債券唯一的風險來源; 
			\item 時間遞減 (time decay) 亦是一種「風險」, 雖然它是可預期且確定性的; 
			\item 時間對債券價格的影響可歸結為「價格逐漸回歸面值」 (pull to par) 的效應; 
			\item 假設所有遠期利率皆為 $5\%$.  
		\end{itemize}
		
		\subsection{總結}
		\begin{itemize}
			\item 我們討論了有息與無息債券; 
			\item 我們學習了現值 (present value), 貼現函數 (discount function), 以及如何將複雜的現金流折現至當下; 
			\item 我們看到, 使現值等於市場價格的貼現率即為到期收益率 (yield to maturity) ; 
			\item 我們了解了當利率上升時, 債券價格會下降; 
			\item 我們探討了票面利率 (coupon rate), 當期收益率 (current yield) 與到期收益率 (YTM) 之間的關係; 
			\item 我們討論了結構 (term structure) 與收益率曲線 (yield curve), 並了解該曲線可等價地用一組即期利率, 遠期利率或零息債券價格來描述; 
			\item 我們看到, 債券價格對利率變動的敏感度可以用一個數值——久期 (duration) ——來量化.  
		\end{itemize}
		
		
	\end{document}

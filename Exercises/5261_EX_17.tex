\documentclass[letterpaper, 9pt]{article}
\linespread{0.85}
\usepackage[utf8]{inputenc}
\usepackage[T1]{fontenc}
\usepackage{amsmath}
\usepackage{amsfonts}
\usepackage{amssymb}
\usepackage{array}
\usepackage{booktabs}
\usepackage{hyperref}
\usepackage{physics}
\usepackage[version=4]{mhchem}
\usepackage{stmaryrd}
\usepackage[dvipsnames]{xcolor}
\colorlet{LightRubineRed}{RubineRed!70}
\colorlet{Mycolor1}{green!10!orange}
\definecolor{Mycolor2}{HTML}{00F9DE}
\usepackage{graphicx}
\usepackage{amsmath}
\usepackage{graphicx}
\usepackage{capt-of}
\usepackage{lipsum}
\usepackage{fancyvrb}
\usepackage{tabularx}
\usepackage{listings}
\usepackage[export]{adjustbox}
\graphicspath{ {./images/} }
\usepackage[utf8]{inputenc}
\usepackage[english]{babel}
\usepackage{float}
\usepackage{lipsum}
\usepackage{graphicx}
\usepackage{float}
\usepackage[margin=0.7in]{geometry}
\usepackage{amsmath}
\usepackage{graphicx}
\usepackage{capt-of}
\usepackage{tcolorbox}
\usepackage{lipsum}
\usepackage{graphicx}
\usepackage{float}
\usepackage{listings}
\usepackage{hyperref} 
\newcommand{\Var}{\mathrm{Var}}
\newcommand{\Cov}{\mathrm{Cov}}
\newcommand{\E}{\mathbb{E}}
\usepackage[normalem]{ulem}
\usepackage{xcolor} % For custom colors
\lstset{
	language=Python,                % Choose the language (e.g., Python, C, R)
	basicstyle=\ttfamily\small, % Font size and type
	keywordstyle=\color{blue},  % Keywords color
	commentstyle=\color{gray},  % Comments color
	stringstyle=\color{red},    % String color
	numbers=left,               % Line numbers
	numberstyle=\tiny\color{gray}, % Line number style
	stepnumber=1,               % Numbering step
	breaklines=true,            % Auto line break
	backgroundcolor=\color{black!5}, % Light gray background
	frame=single,               % Frame around the code
}
\usepackage{float}
\usepackage[]{amsthm} %lets us use \begin{proof}
	\usepackage[]{amssymb} %gives us the character \varnothing
	
	\title{Exercise 17, STAT 5261}
	\author{Zongyi Liu}
	\date{Wed, Oct 15, 2025}
	
	\begin{document}
		\maketitle
		
		
		\section{Question 1}
		What is the beta of a portfolio if $E\left(R_{P}\right)=16 \%, \mu_{f}=5.5 \%$, and $E\left(R_{M}\right)=11 \%$?
		
	
	\textbf{Answer}
 	\vspace*{0.1\textheight}
		
			\section{Question 2}
			
		 Suppose that the risk-free rate of interest is 0.03 and the expected rate of return on the market portfolio is 0.14. The standard deviation of the market portfolio is 0.12.\\
		 (a) According to the CAPM, what is the efficient way to invest with an expected rate of return of 0.11?\\
		 (b) What is the risk (standard deviation) of the portfolio in part (a)?
		 
		 \textbf{Answer}
		 \vspace*{0.1\textheight}
		 
		 	\section{Question 3}
		 Suppose that the risk-free interest rate is 0.023 , that the expected return on the market portfolio is $\mu_{M}=0.10$, and that the volatility of the market portfolio is $\sigma_{M}=0.12$.\\
		 (a) What is the expected return on an efficient portfolio with $\sigma_{R}=0.05$?\\
		 (b) Stock A returns have a covariance of 0.004 with market returns. What is the beta of Stock A?\\
		 (c) Stock B has beta equal to 1.5 and $\sigma_{\epsilon}=0.08$. Stock C has beta equal to 1.8 and $\sigma_{\epsilon}=0.10$.\\
		 i. What is the expected return of a portfolio that is one-half Stock B and one-half Stock C?\\
		 ii. What is the volatility of a portfolio that is one-half Stock B and one-half Stock C? Assume that the $\epsilon$ s of Stocks B and C are independent.
		 
		 
		 \textbf{Answer}
		\vspace*{0.1\textheight}
		
			\section{Question 4}
		Show that equation (17.16) follows from equation (7.8).
		
		\textbf{Answer}
		\vspace*{0.1\textheight}
		
			\section{Question 5}
		 True or false: The CAPM implies that investors demand a higher return to hold more volatile securities. Explain your answer.
	
	\textbf{Answer}
	\vspace*{0.1\textheight}
	
	\section{Question 6}
	Suppose that the riskless rate of return is $4 \%$ and the expected market return is $12 \%$. The standard deviation of the market return is $11 \%$. Suppose as well that the covariance of the return on Stock A with the market return is $165 \%^{2}.^{9}$\\
	(a) What is the beta of Stock A?\\
	(b) What is the expected return on Stock A?\\
	(c) If the variance of the return on Stock A is $220 \%^{2}$, what percentage of this variance is due to market risk?
	
	
	\textbf{Answer}
	\vspace*{0.1\textheight}
	
	\section{Question 7}
	
	Suppose there are three risky assets with the following betas and $\sigma_{\epsilon_{j}}^{2}$.

\begin{center}
\begin{tabular}{l|c|c}
	$j$ & $\beta_{j}$ & $\sigma_{\epsilon_{j}}^{2}$ \\
	\hline
	1 & 0.9 & 0.010 \\
	2 & 1.1 & 0.015 \\
	3 & 0.6 & 0.011 \\
\end{tabular}
\end{center}

Suppose also that the variance of $R_{M t}-\mu_{f t}$ is 0.014.\\
(a) What is the beta of an equally weighted portfolio of these three assets?\\
(b) What is the variance of the excess return on the equally weighted portfolio?\\
(c) What proportion of the total risk of asset 1 is due to market risk?


\textbf{Answer}
\vspace*{0.2\textheight}

\section{Question 8}
Suppose there are two risky assets, call them C and D. The tangency portfolio is $60 \% \mathrm{C}$ and $40 \% \mathrm{D}$. The expected yearly returns are $4 \%$ and $6 \%$ for assets C and D. The standard deviations of the yearly returns are $10 \%$ and $18 \%$ for C and D and the correlation between the returns on C and D is 0.5. The risk-free yearly rate is $1.2 \%$.\\
(a) What is the expected yearly return on the tangency portfolio?\\
(b) What is the standard deviation of the yearly return on the tangency portfolio?\\
(c) If you want an efficient portfolio with a standard deviation of the yearly return equal to $3 \%$, what proportion of your equity should be in the risk-free asset? If there is more than one solution, use the portfolio with the higher expected yearly return.\\
(d) If you want an efficient portfolio with an expected yearly return equal to $7 \%$, what proportions of your equity should be in asset C, asset D, and the risk-free asset?\\

\textbf{Answer}
\vspace*{0.3\textheight}
\section{Question 9}
What is the beta of a portfolio if the expected return on the portfolio is $E\left(R_{P}\right)=15 \%$, the risk-free rate is $\mu_{f}=6 \%$, and the expected return on the market is $E\left(R_{M}\right)=12 \%$? Make the usual CAPM assumptions including that the portfolio alpha is zero.

\textbf{Answer}
\clearpage

\section{Question 10}
Suppose that the risk-free rate of interest is 0.07 and the expected rate of return on the market portfolio is 0.14. The standard deviation of the market portfolio is 0.12.\\
(a) According to the CAPM, what is the efficient way to invest with an expected rate of return of 0.11?\\
(b) What is the risk (standard deviation) of the portfolio in part (a)?

\textbf{Answer}

\vspace*{0.2\textheight}

\section{Question 11}
Suppose there are three risky assets with the following betas and $\sigma_{\epsilon_{j}}^{2}$ when regressed on the market portfolio.

\begin{center}
	\begin{tabular}{l|c|c}
		$j$ & $\beta_{j}$ & $\sigma_{\epsilon_{j}}^{2}$ \\
		\hline
		1 & 0.7 & 0.010 \\
		2 & 0.8 & 0.025 \\
		3 & 0.6 & 0.012 \\
	\end{tabular}
\end{center}

Assume $\epsilon_{1}, \epsilon_{2}$, and $\epsilon_{3}$ are uncorrelated. Suppose also that the variance of $R_{M}-\mu_{f}$ is 0.02.\\
(a) What is the beta of an equally weighted portfolio of these three assets?\\
(b) What is the variance of the excess return on the equally weighted portfolio?\\
(c) What proportion of the total risk of asset 1 is due to market risk?

		\textbf{Answer}
		\vspace*{0.2\textheight}
		
		\section{Question 12}
		As an analyst, you have constructed 2 possible portfolios. Both portfolios have the same beta and expected return, but portfolio 1 was constructed with only technology companies whereas portfolio 2 was constructed using technology, healthcare, energy, consumer products, and metals and mining companies. Should you be impartial to which portfolio you invest in? Explain why or why not.
		
			\textbf{Answer}
		

\clearpage
	\end{document}
	

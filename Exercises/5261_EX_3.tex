\documentclass[letterpaper, 9pt]{article}
\linespread{0.85}
\usepackage[utf8]{inputenc}
\usepackage[T1]{fontenc}
\usepackage{amsmath}
\usepackage{amsfonts}
\usepackage{amssymb}
\usepackage{array}
\usepackage{booktabs}
\usepackage{hyperref}
\usepackage{physics}
\usepackage[version=4]{mhchem}
\usepackage{stmaryrd}
\usepackage[dvipsnames]{xcolor}
\colorlet{LightRubineRed}{RubineRed!70}
\colorlet{Mycolor1}{green!10!orange}
\definecolor{Mycolor2}{HTML}{00F9DE}
\usepackage{graphicx}
\usepackage{amsmath}
\usepackage{graphicx}
\usepackage{capt-of}
\usepackage{lipsum}
\usepackage{fancyvrb}
\usepackage{tabularx}
\usepackage{listings}
\usepackage[export]{adjustbox}
\graphicspath{ {./images/} }
\usepackage[utf8]{inputenc}
\usepackage[english]{babel}
\usepackage{float}
\usepackage{lipsum}
\usepackage{graphicx}
\usepackage{float}
\usepackage[margin=0.7in]{geometry}
\usepackage{amsmath}
\usepackage{graphicx}
\usepackage{capt-of}
\usepackage{tcolorbox}
\usepackage{lipsum}
\usepackage{graphicx}
\usepackage{float}
\usepackage{listings}
\usepackage{hyperref} 
\newcommand{\Var}{\mathrm{Var}}
\newcommand{\Cov}{\mathrm{Cov}}
\newcommand{\E}{\mathbb{E}}
\usepackage[normalem]{ulem}
\usepackage{xcolor} % For custom colors
\lstset{
	language=Python,                % Choose the language (e.g., Python, C, R)
	basicstyle=\ttfamily\small, % Font size and type
	keywordstyle=\color{blue},  % Keywords color
	commentstyle=\color{gray},  % Comments color
	stringstyle=\color{red},    % String color
	numbers=left,               % Line numbers
	numberstyle=\tiny\color{gray}, % Line number style
	stepnumber=1,               % Numbering step
	breaklines=true,            % Auto line break
	backgroundcolor=\color{black!5}, % Light gray background
	frame=single,               % Frame around the code
}
\usepackage{float}
\usepackage[]{amsthm} %lets us use \begin{proof}
	\usepackage[]{amssymb} %gives us the character \varnothing
	
	\title{Exercise 3, STAT 5261}
	\author{Zongyi Liu}
	\date{Wed, Oct 15, 2025}
	
	\begin{document}
		\maketitle
		
		
		\section{Question 1}
		
		Suppose that the forward rate is $r(t)=0.028+0.00042 t$.\\
		(a) What is the yield to maturity of a bond maturing in 20 years?\\
		(b) What is the price of a par $\$ 1,000$ zero-coupon bond maturing in 15 years?
		
		\textbf{Answer}
		\vspace*{0.2\textheight}
		
		\section{Question 2}
		
		Suppose that the forward rate is $r(t)=0.04+0.0002 t-0.00003 t^{2}$.\\
		(a) What is the yield to maturity of a bond maturing in 8 years?\\
		(b) What is the price of a par $\$ 1,000$ zero-coupon bond maturing in 5 years?\\
		(c) Plot the forward rate and the yield curve. Describe the two curves. Which are convex and which are concave? How do they differ?\\
		(d) Suppose you buy a 10 -year zero-coupon bond and sell it after 1 year. What will be the return if the forward rate does not change during that year?
		
		\textbf{Answer}
		\vspace*{0.2\textheight}
		
		\section{Question 3}
		A coupon bond has a coupon rate of $3 \%$ and a current yield of $2.8 \%$.\\
		(a) Is the bond selling above or below par? Why or why not?\\
		(b) Is the yield to maturity above or below $2.8 \%$? Why or why not?
		
		\textbf{Answer}
		
		\vspace*{0.2\textheight}
		
		\section{Question 4}
		Suppose that the forward rate is $r(t)=0.032+0.001 t+0.0002 t^{2}$.\\
		(a) What is the 5 -year continuously compounded spot rate?\\
		(b) What is the price of a zero-coupon bond that matures in 5 years?
		
		\textbf{Answer}
		\vspace*{0.2\textheight}
		
		\section{Question 5}
		The $1 / 2-$, $1-, 1.5-$, and 2-year semiannually compounded spot rates are $0.025,0.028,0.032$, and 0.033 , respectively. A par $\$ 1,000$ coupon bond matures in 2 years and has semiannual coupon payments of $\$ 35$. What is the price of this bond?
		
		
		\textbf{Answer}
		\vspace*{0.3\textheight}
		
		\section{Question 6}
		Verify the following equality:
		
		$$
		\sum_{t=1}^{2 T} \frac{C}{(1+r)^{t}}+\frac{\mathrm{PAR}}{(1+r)^{2 T}}=\frac{C}{r}+\left\{\mathrm{PAR}-\frac{C}{r}\right\}(1+r)^{-2 T}
		$$
		
		
		\textbf{Answer}
		\vspace*{0.2\textheight}
		
		\section{Question 7}
		One year ago a par $\$ 1,00020$-year coupon bond with semiannual coupon payments was issued. The annual interest rate (that is, the coupon rate) at that time was $8.5 \%$. Now, a year later, the annual interest rate is $7.6 \%$.\\
		(a) What are the coupon payments?\\
		(b) What is the bond worth now? Assume that the second coupon payment was just received, so the bondholder receives an additional 38 coupon payments, the next one in 6 months.\\
		(c) What would the bond be worth if instead the second payment were just about to be received?
		
		\textbf{Answer}
		\vspace*{0.2\textheight}
		
		\section{Question 8}
		A par $\$ 1,000$ zero-coupon bond that matures in 5 years sells for $\$ 828$. Assume that there is a constant continuously compounded forward rate $r$.\\
		(a) What is $r$?\\
		(b) Suppose that 1 year later the forward rate $r$ is still constant but has changed to be 0.042. Now what is the price of the bond?\\
		(c) If you bought the bond for the original price of $\$ 828$ and sold it 1 year later for the price computed in part (b), then what is the net return?
		
		\textbf{Answer}
		\vspace*{0.2\textheight}
		
		\section{Question 9}
		A coupon bond with a par value of $\$ 1,000$ and a 10-year maturity pays semiannual coupons of $\$ 21$.\\
		(a) Suppose the yield for this bond is $4 \%$ per year compounded semiannually. What is the price of the bond?\\
		(b) Is the bond selling above or below par value? Why?
		
		\textbf{Answer}
		\vspace*{0.2\textheight}
		
		\section{Question 10}
		Suppose that a coupon bond with a par value of $\$ 1,000$ and a maturity of 7 years is selling for $\$ 1,040$. The semiannual coupon payments are $\$ 23$.\\
		(a) Find the yield to maturity of this bond.\\
		(b) What is the current yield on this bond?\\
		(c) Is the yield to maturity less or greater than the current yield? Why?
		
		\textbf{Answer}
		\vspace*{0.3\textheight}
		
		\section{Question 11}
		Suppose that the continuous forward rate is $r(t)=0.033+0.0012 t$. What is the current value of a par $\$ 100$ zero-coupon bond with a maturity of 15 years?
		
		\textbf{Answer}
		\vspace*{0.2\textheight}
		
		\section{Question 12}
		Suppose the continuous forward rate is $r(t)=0.04+0.001 t$ when a 8 -year zero coupon bond is purchased. Six months later the forward rate is $r(t)= 0.03+0.0013 t$ and bond is sold. What is the return?
		
		\textbf{Answer}
		\vspace*{0.2\textheight}
		
		\section{Question 13}
		Suppose that the continuous forward rate is $r(t)=0.03+0.001 t- 0.00021(t-10)_{+}$. What is the yield to maturity on a 20 -year zero-coupon bond? Here $x_{+}$is the positive part function defined by
		
		$$
		x_{+}= \begin{cases}x, & x>0 \\ 0, & x \leq 0\end{cases}
		$$
		
		\textbf{Answer}
		\vspace*{0.2\textheight}
		
		\section{Question 14}
		An investor is considering the purchase of zero-coupon bonds with maturities of one, three, or 5 years. Currently the spot rates for 1-, 2-, 3-, 4-, and 5-year zero-coupon bonds are, respectively, $0.031,0.035,0.04,0.042$, and 0.043 per year with semiannual compounding. A financial analyst has advised this investor that interest rates will increase during the next year and the analyst expects all spot rates to increase by the amount 0.005 , so that the 1-year spot rate will become 0.036 , and so forth. The investor plans to sell the bond at the end of 1 year and wants the greatest return for the year. This problem does the bond math to see which maturity, 1, 3, or 5 years, will give the best return under two scenarios: interest rates are unchanged and interest rates increase as forecast by the analyst.\\
		(a) What are the current prices of 1-, 3-, and 5 -year zero-coupon bonds with par values of $\$ 1,000$?\\
		(b) What will be the prices of these bonds 1 year from now if spot rates remain unchanged?\\
		(c) What will be the prices of these bonds 1 year from now if spot rates each increase by 0.005?\\
		(d) If the analyst is correct that spot rates will increase by 0.005 in 1 year, which maturity, 1,3 , or 5 years, will give the investor the greatest return when the bond is sold after 1 year? Justify your answer.\\
		(e) If instead the analyst is incorrect and spot rates remain unchanged, then which maturity, 1, 3, or 5 years, earns the highest return when the bond is sold after 1 year? Justify your answer.\\
		(f) The analyst also said that if the spot rates remain unchanged, then the bond with the highest spot rate will earn the greatest 1 -year return. Is this correct? Why?\\
		(Hint: Be aware that a bond will not have the same maturity in 1 year as it has now, so the spot rate that applies to that bond will change.)
		
		\textbf{Answer}
		\clearpage
		
		\section{Question 15}
		Suppose that a bond pays a cash flow $C_{i}$ at time $T_{i}$ for $i=1, \ldots, N$. Then the net present value (NPV) of cash flow $C_{i}$ is
		
		$$
		\mathrm{NPV}_{i}=C_{i} \exp \left(-T_{i} y_{T_{i}}\right)
		$$
		
		Define the weights
		
		$$
		\omega_{i}=\frac{\mathrm{NPV}_{i}}{\sum_{j=1}^{N} \mathrm{NPV}_{j}}
		$$
		
		and define the duration of the bond to be
		
		$$
		\mathrm{DUR}=\sum_{i=1}^{N} \omega_{i} T_{i}
		$$
		
		which is the weighted average of the times of the cash flows. Show that
		
		$$
		\left.\frac{d}{d \delta} \sum_{i=1}^{N} C_{i} \exp \left\{-T_{i}\left(y_{T_{i}}+\delta\right)\right\}\right|_{\delta=0}=-\mathrm{DUR} \sum_{i=1}^{N} C_{i} \exp \left\{-T_{i} y_{T_{i}}\right\}
		$$
		
		and use this result to verify Eq. (3.31).
		
		\textbf{Answer}
		\vspace*{0.2\textheight}
		
		\section{Question 16}
		Assume that the yield curve is $Y_{T}=0.04+0.001 T$.\\
		(a) What is the price of a par- $\$ 1,000$ zero-coupon bond with a maturity of 10 years?\\
		(b) Suppose you buy this bond. If 1 year later the yield curve is $Y_{T}= 0.042+0.001 T$, then what will be the net return on the bond?
		
		\textbf{Answer}
		\clearpage
		
		\section{Question 17}
		A coupon bond has a coupon rate of $3 \%$ and a current yield of $2.8 \%$.\\
		(a) Is the bond selling above or below par? Why or why not?\\
		(b) Is the yield to maturity above or below $2.8 \%$? Why or why not?
		
		\textbf{Answer}
		\vspace*{0.2\textheight}
		
		\section{Question 18}
		Suppose that the forward rate is $r(t)=0.03+0.001 t+0.0002 t^{2}$\\
		(a) What is the 5 -year spot rate?\\
		(b) What is the price of a zero-coupon bond that matures in 5 years?
		
		\textbf{Answer}
		\vspace*{0.2\textheight}
		
		\section{Question 19}
		The $1 / 2-, 1-, 1.5-$, and 2 -year spot rates are $0.025,0.029,0.031$, and 0.035 , respectively. A par $\$ 1,000$ coupon bond matures in 2 years and has semiannual coupon payments of $\$ 35$. What is the price of this bond?
		
		\textbf{Answer}
		\clearpage
		
		\section{Question 20}
		Par $\$ 1,000$ zero-coupon bonds of maturities of $0.5-, 1-, 1.5-$, and $2-$years are selling at $\$ 980.39, \$ 957.41, \$ 923.18$, and $\$ 888.489$, respectively.\\
		(a) Find the $0.5-$, $1-$, $1.5-$, and $2 -$year semiannual spot rates.\\
		(b) A par $\$ 1,000$ coupon bond has a maturity of 2 years. The semiannual coupon payment is $\$ 21$. What is the price of this bond?
		
		\textbf{Answer}
		\vspace*{0.2\textheight}
		
		\section{Question 21}
		A par $\$ 1,000$ bond matures in 4 years and pays semiannual coupon payments of $\$ 25$. The price of the bond is $\$ 1,015$. What is the semiannual yield to maturity of this bond?
		
		\textbf{Answer}
		\vspace*{0.2\textheight}
		
		\section{Question 22}
		A coupon bond matures in 4 years. Its par is $\$ 1,000$ and it makes eight coupon payments of $\$ 21$, one every one-half year. The continuously compounded forward rate is $
		r(t)=0.022+0.005 t-0.004 t^{2}+0.0003 t^{3}.$
		
		(a) Find the price of the bond.\\
		(b) Find the duration of this bond.
		
		
		\textbf{Answer}
		\vspace*{0.2\textheight}
		
		\clearpage
	\end{document}
	
